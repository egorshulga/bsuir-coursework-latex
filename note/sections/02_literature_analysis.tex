\section{Анализ литературных источников}
\label{sec:literature_analysis}

\subsection{Системы управления базами данных}

База данных -- представленная в объективной форме совокупность самостоятельных материалов, систематизированных таким образом, чтобы эти материалы могли быть найдены и обработаны с помощью ЭВМ \cite{wiki_db}. Выделяют следующие отличительные признаки баз данных:
\begin{itemize}
	\item БД хранится и обрабатывается в вычислительной системе.
	\item Данные в БД логически структурированы (систематизированы) с целью обеспечения возможности их эффективного поиска и обработки в вычислительной системе. Структурированность подразумевает явное выделение составных частей (элементов), связей между ними, а также типизацию элементов и связей, при которой с типом элемента (связи) соотносится определённая семантика и допустимые операции.
	\item БД включает схему, или метаданные, описывающие логическую структуру БД в формальном виде. Схема включает в себя описания содержания, структуры и ограничений целостности, используемые для создания и поддержки базы данных.
\end{itemize}

Существует огромное количество разновидностей баз данных, отличающихся по различным критериям. Например, по среде постоянного хранения базы данных можно классифицировать следующим образом:
\begin{itemize}
	\item Традиционная БД: средой постоянного хранения является периферийная энергонезависимая память (вторичная память) — как правило жёсткий диск.
	\item In-memory database: все данные на стадии исполнения находятся в оперативной памяти.
	\item Tertiary database: средой постоянного хранения является отсоединяемое от сервера устройство массового хранения (третичная память), как правило на основе магнитных лент или оптических дисков. Во вторичной памяти сервера хранится лишь каталог данных третичной памяти, файловый кеш и данные для текущей обработки; загрузка же самих данных требует специальной процедуры.
\end{itemize}

По степени распределенности предлагается следующая классификация: 
\begin{itemize}
	\item Централизованные БД: полностью поддерживаются на одном компьютере.
	\item Распределенные БД: составные части размещаются в различных узлах компьютерной сети. 
\end{itemize}

По модели данных базы данных можно классифицировать следующим образом \cite{kulikov_db_workbook}:
\begin{itemize}
	\item Картотека: упорядоченное (по алфавиту, дате и т.п.) собрание данных в виде записей (<<карт>>), каждая из которых предоставляет сведения о каком-то объекте базы данных.
	\item Сетевые БД: каждый элемент связан с некоторыми другими.
	\item Иерархические БД: построены на основе какой-либо иерархической структуры. 
	\item Реляционные БД: основаны на теоретико-множественной реляционной даталогической модели (предложена доктором Эдгаром Коддом в 1970 году). Все данные представлены в виде (связанных между собой) таблиц, разбитых на строки и столбцы.
	\item Многомерные БД: предназначены для обработки данных из различных источников и временных данных. Могут строиться на основе реляционных БД или на основе других более сложных хранилищ.
	\item Объектно-ориентированные БД: данные оформлены в виде моделей объектов, включающих прикладные программы, которые управляются внешними событиями.
	\item Дедуктивные БД: состоит из двух частей: экстенциональной (содержащей факты) и интенциональной (содержащей правила для логического вывода новых фактов).
	\item NoSQL БД: описание схемы данных в случае использования NoSQL-решений может осуществляться через использование различных структур данных: хеш-таблиц, деревьев и т.д.
\end{itemize}

В данной курсовой работе, в соответствие с поставленной задачей и материалом изученного курса Баз Данных, используется реляционная СУБД.

\subsection{Язык SQL}

В 1970 году Эдгар Кодд публикует результаты своих исследований в области реляционной теории \cite{codd}. После этого IBM начинает разработку системы баз данных System R \cite{wiki_codd}. Одновременно велась разработка языка программирования для осуществления операций над данными; первая его версия, разработанная в начале 1970-х, имела название SEQUEL (Structured English Query Language). Позже, по причинам, не связанным с программированием, название было изменено на SQL (Structured Query Language), но из первоначального названия становится понятна цель, которую преследовали разработчики: создать язык управления данными, который бы строился по принципам естественных языков и был понятен человеку.

Уже в 1980-х было выпущено несколько систем управления базами данных, они конкурировали между собой, и встал вопрос о стандартизации языка управления данными. И уже в 1986 году была принята первая версия стандарта SQL (SQL-86). Следующие major версии были приняты в 1992, 1999, 2003, 2008, 2011 годах. 

Несмотря на стандарты, все современные СУБД не совместимы между собой. Такие их аспекты, как, например, обработка даты и времени, строк, регистрозависимость, отличаются и очень часто не дают возможности переноса кода без каких-либо модификаций. Основные причины для существования этой несовместимости следующие \cite{wiki_sql}:
\begin{itemize}
	\item Сложность и объемность стандартов SQL: разработчики стремятся поддерживать стандарты, но из-за его размером у них не получается поддерживать его целиком.
	\item Стандарт не определяет некоторые важные аспекты поведения СУБД, такие как индексы, хранение данных.
	\item Стандарт точно определяет синтаксис языка, но не семантику его конструкций.
	\item Часто производители отказываются от поддержки некоторых частей новых стандартов ради обратной совместимости.
	\item Производителям выгодно, чтобы пользователи не могли легко сменить СУБД.
	\item Пользователи часто выбирают более производительные решения, чем тем, которые в большей степени реализуют стандарт.
\end{itemize}

Всё это привело к тому, что существует огромное число диалектов языка SQL, каждый из которых предоставляет разные возможности. Например, управление СУБД Microsoft SQL Server осуществляется с помощью диалекта Transact SQL (T-SQL), Oracle для своей СУБД реализовало объектно-ориентированный язык программирования PL/SQL.

Тем не менее, несмотря на все недостатки и сложности, системы управления базами данных и язык SQL (все его диалекты) всё равно продолжают использоваться практически везде, что приводит к тому, что этот язык продолжает занимать первую позицию в различных рейтингах языков программирования по востребованности и не только. 

\subsection{Проектирование баз данных}

Модель базы данных -- описание базы данных с помощью  определенного (в т.ч. графического) языка на некотором уровне абстракции.

Основные задачи проектирования баз данных \cite{wiki_db_modelling}:
\begin{itemize}
	\item Обеспечение хранения в БД всей необходимой информации.
	\item Обеспечение возможности получения данных по всем необходимым запросам.
	\item Сокращение избыточности и дублирования данных.
	\item Обеспечение целостности базы данных.
\end{itemize}

Выделяют следующие уровни моделирования \cite{kulikov_db_workbook}:
\begin{itemize}
	\item инфологический уровень: описание предметной области без привязок к каким-либо средствам реализации: языкам программирования, СУБД, и т.д;
	\item даталогический уровень: модель предметной области в привязке к средствам реализации;
	\item физический уровень: описывает конкретные таблицы, связи, индексы, методы хранения, настройки производительности, безопасности и т.д.
\end{itemize}

При ошибках моделирования могут возникать аномалии операций с БД. Аномалия -- противоречие между моделью предметной области и моделью данных, поддерживаемой средствами конкретной СУБД \cite{kulikov_db_workbook}.

Выделяют следующие виды аномалий:
\begin{itemize}
	\item Аномалия вставки: при добавлении данных, часть которых у нас отсутствует, мы вынуждены или не выполнять добавление или подставлять пустые или фиктивные данные.
	\item Аномалия обновления – при обновлении данных мы вынуждены обновлять много строк и рискуем часть строк «забыть обновить».
	\item Аномалия удаления – при удалении части данных мы теряем другую часть, которую не надо было удалять.
\end{itemize}

Для устранения аномалий существует процесс нормализации БД, который основан на теории зависимостей. Приведем некоторые понятия этой теории.

Отношение R степени n -- подмножество декартового произведения множеств $D_1, D_2,.. D_n (n \ge 1)$. Исходные множества $D_1, D_2,.. D_n (n \ge 1)$ называются доменами.

Ключ -- атрибут или совокупность атрибутов отношения, обладающие некоторыми специфическими свойствами, зависящими от ключа.

Первичный ключ -- минимальное множество атрибутов,
являющееся подмножеством заголовка данного отношения, составное значение которых уникально определяет кортеж отношения.

Возможный ключ -- поле или совокупность полей с уникальными значениями, кандидат в первичные ключи.

Альтернативный ключ -- поле или совокупность полей с
уникальными значениями, не выбранные в качестве первичного ключа.

Внешний ключ -- поле таблицы, предназначенное для хранения значения первичного ключа другой таблицы с целью организации связи между этими таблицами.

Связь -- ассоциация, установленная между двумя и более отношениями.

Идентифицирующая связь определяет ситуацию, когда запись в дочерней таблице обязана быть связана с записью в родительской таблице.

Неидентифицирующая связь определяет ситуацию, когда запись в дочерней таблице может быть НЕ связана с записью в родительской таблице.

Ссылочная целостность -- необходимое качество реляционной БД, заключающееся в отсутствии в любом её отношении внешних ключей, ссылающихся на несуществующие кортежи.

Функциональная зависимость: если даны два атрибута X и Y некоторого отношения, то Y функционально зависит от X, если в любой момент времени каждому значению X соответствует ровно одно значение Y.

Функциональная зависимость $X \rightarrow Y$ является полной, если Y не зависит функционально от любого подмножества X.

Функциональная зависимость $X \rightarrow Y$ является частичной, если Y зависит функционально от некоторого подмножества X.

Функциональная зависимость $X \rightarrow Y$ является транзитивной, если существуют зависимости $X \rightarrow Z$ и $Z \rightarrow Y$, но отсутствует прямая зависимость $X \rightarrow Y$.

Многозначная зависимость $X \rightarrow\rightarrow Y | Z$ существует в том и только в том случае, если множество значений Y, соответствующее паре значений X и Z, зависит только от X и не зависит от Z (то есть для каждого значения атрибута X существует множество соответствующих значений атрибута Y).

Многозначная зависимость называется тривиальной, если она содержит хотя бы одну функциональную зависимость.

Нормализация -- группировка и/или распределение атрибутов по отношениям с целью устранения аномалий операций с БД, обеспечения целостности данных и оптимизации модели БД.

Отношение находится в первой нормальной форме, если все его атрибуты являются атомарными. Атрибут считается атомарным, если в предметной области не существует операции, для выполнения которой понадобилось бы извлечь часть атрибута.

Отношение находится во второй нормальной форме, если оно находится в первой нормальной форме, и при этом  любой атрибут, не входящий в состав ПК, функционально полно зависит от ПК.

Отношение находится в третьей нормальной форме, если оно находится во второй нормальной форме, и при этом любой его неключевой атрибут нетранзитивно зависит от первичного ключа.

Отношение находится в нормальной форме Бойса-Кодда, если детерминанты всех функциональных зависимостей являются потенциальными ключами.

Детерминант функциональной зависимости -- атрибут, от которого функционально-полно зависит некоторый другой атрибут.

Отношение находится в четвертой нормальной форме, если оно находится в третьей нормальной форме, и не содержит нетривиальных многозначных зависимостей (т.е. все его зависимости являются функциональными от ключа).

Отношение находится в пятой нормальной форме, если оно находится в четвертой нормальной форме, и любая многозначная зависимость соединения в нём является тривиальной.

Отношение находится в доменно-ключевой нормальной форме, если его структура не допускает аномалий вставки, обновления и удаления, а также позволяет контролировать значения атрибутов там, где это имеет смысл.

Отношение находится в шестой нормальной форме, если оно находится в пятой нормальной форме, и его проекции не приводят к потере темпоральных данных.

Денормализация - процесс приведения отношения к состоянию, нарушающему те или иные нормальные формы. В основном применяется для оптимизации некоторых видов запросов.

Результатом моделирования базы данных (на даталогическом уровне) становится модель базы данных, которая, как правило, представлена в виде UML диаграмм. Многие CASE средства, которые позволяют проводить этот этап моделирования, предоставляют средства для перехода на физический уровень моделирования: они позволяют каким-либо образом осуществить развертывание базы данных на выбранной СУБД. Обычно развертывание баз данных осуществляется с помощью специальных скриптов, которые и предоставляют такие CASE средства.
