\sectioncentered*{Заключение}
\addcontentsline{toc}{section}{Заключение}

В данной курсовой работе был рассмотрен вопрос проектирования базы данных на даталогическом уровне. 

В рамках работы была спроектирована база данных для предметной области автоматизации некоторых аспектов учебной деятельности университета. Реализовано разделение ролей участников учебного процесса с  возможностью их многократного назначения и снятия. 

Для управляющих различных элементов структуры университетов \\предусмотрены разнообразные возможности, например: управление новостями, участниками и элементами учебного процесса и т.д. Для преподавателей предусмотрены возможности управления материалами, предназначенными для студентов, управление списками студентов, учёт прогресса сдачи лабораторных и других заданий, учёт посещаемости студентами занятий. Для студентов в базе данных предусмотрена возможность учёта всех видов заданий, деятельностей, итогового контроля знаний по предметам, сохранение результатов контроля для просмотра их в будущем в виде электронной зачётной книжки. 

Для всех участников учебного процесса предусмотрена возможность формирования расписания в форме ежедневника. Кроме того, в базе данных предусмотрена возможность реализации системы личных сообщений и оповещений пользователей о различных изменениях: от преподавателей, от работников деканата; о предметах экзаменах и других видах деятельности. Предусмотрена подсистема управления файлами, прикрепляемых к сообщениям, условиям заданий, новостям и т.д.

В качестве результата процесса проектирования была получена модель базы данных, которая затем была развернута и заполнена тестовыми данными. В дальнейшем спроектированную базу данных можно использовать для разработки и реализации информационной системы, позволяющей оцифровать многие деятельности участников учебного процесса. 
