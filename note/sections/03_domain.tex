\section{Обзор предметной области}
\label{sec:domain}

\subsection{Обзор аналогов}
\label{sec:domain:analogues}

В качестве предметной области была выбрана автоматизация работы преподавателя не случайно. В настоящее время на постсоветском пространстве существует огромное число высших учебных заведений. Некоторое время назад процессы, происходящие в таких учреждениях не были оцифрованы и автоматизированы вовсе. Лишь недавно начали появляться информационные системы, решающие те или иные задачи. Однако, каждый ВУЗ занимается этой проблемой самостоятельно и решает ее, насколько хватает квалифицированных специалистов, ресурсов и, самое главное, желания внедрять что-то инновационное со стороны руководства и участников учебного процесса этих ВУЗов. 

Универсальной шкалы, позволяющей оценить удобство следования по всем деятельностям учебного процесса в разных ВУЗах, нет. Однако, можно попытаться это оценить по некоторым вторичным признакам, таким как, например, соответствие сайта ВУЗа современным представлениям об удобстве пользования (можно легко отличить сайты, сделанные за последние несколько лет, от сайтов, сделанных более десяти лет назад: со временем меняются и представления об удобстве пользования, и технологии, позволяющие его достичь), отзывах тех людей, которые участвуют в процессе обучения (то есть студентов, преподавателей) и некоторым другим.

Например, Белорусский государственный университет информатики, как ведущий специализированный университет в области информационных технологий на постсоветском пространстве, обладает достаточным числом квалифицированных специалистов (поскольку обучает их), чтобы автоматизировать многие части учебного процесса. Официальный сайт сделан с применением современных технологий, информация, приведенная на нём, актуальна. Для пользования предоставляется система расписания: занятий и экзаменов для студентов и преподавателей, причём предоставляется возможность не только просматривать расписание, находясь на сайте, но реализован сетевой программный интерфейс (API -- application programming interface) для получения актуальных данных о преподавателях, группах и расписании для них в формате XML. Помимо этого, наиболее близкий к информационным технологиям факультет Компьютерных систем и сетей (КСиС) имеет в своём распоряжении портал факультета, на котором присутствует актуальная информация о событиях, происходящих на факультете, куда выкладываются расписания учебного процесса, списки групп, рейтинговые списки. Однако, по-прежнему существует и ряд проблем. Кроме факультета КСиС, своих порталов не имеет ни один из других факультетов, поэтому вся коммуникация со студентами может происходить только очно, а некоторые неформальные виды коммуникации -- через группы факультетов в социальных сетях. Кафедры не имеют удобной для студентов доски объявлений (строго говоря, такая доска есть, но она находится настолько глубоко внутри сайта университета, что большая часть студентов о ней и не знает). Для того, чтобы внести любое малейшее изменение в расписание, преподавателю нужно обращаться в диспетчерскую службу: нет возможности сделать это автоматически. Но несмотря на все недостатки, БГУИР все равно обгоняет многие ВУЗы по степени информатизации и автоматизации учебного процесса.

Сайт ведущего университета страны общего профиля -- Белорусского государственного университета -- еще только год назад являлся не соответствующим времени. Безусловно, за год была произведена большая работа, он был переписан с нуля, его удобство пользования и актуальность информации на нём значительно повысились. Еще одним достоинством информационной системы БГУ является то, что для всех факультетов реализованы отдельные порталы. Однако, очень большим недостатком является то, что в этом университете не реализована единая система управления расписанием. Все факультеты решают задачу составления расписания и доставки его студентам и преподавателям по-своему. Есть факультеты, которые предоставляют более-менее удобные таблицы с расписанием, есть факультеты, которые предоставляют мобильное приложение с расписанием (но только с расписанием этого факультета), однако, есть факультеты, которые предоставляют расписание в виде неизменяемого и недоступного для автоматизированного разбора документе (например, в виде PDF документа). Более того, некоторые кафедры вообще не выкладывают своё расписание на сайт факультета, и чтобы студенты и преподаватели могли с ним ознакомиться, им нужно приходить к доске объявлений своей кафедры. В итоге, нет ни программного интерфейса для доступа к расписанию, ни даже унифицированной системы расписания, что значительно затрудняет создание любых приложений для работы с ним. И эту проблему в данном ВУЗе можно решить только одним способом: предложить систему, форматы данных и программные интерфейсы и каким-либо образом убедить их использовать. Но процесс информатизации не стоит на месте, и стоит ожидать, что уже скоро такая унификация произойдёт.

В качестве еще одного примера можно привести информационную систему Белорусского государственного университета транспорта (г. Гомель). Еще год назад с сайтом этого университета было много проблем: сайт находился в черных списках по подозрению в распространении вредоносного программного обеспечения, поэтому и браузеры, и антивирусы не позволяли просто открыть этот сайт, а требовали постоянных подтверждений; расписания не выкладывались на сайт университета ни в каком виде; требуемую информацию сложно было найти из-за организации интерфейсов. В настоящее время проблемы начинают решаться, например, все факультеты выкладывают своё расписание, однако все в разных форматах и в виде документов, которые сложно анализировать с помощью какого-либо программного обеспечения. Данный университет также нуждается во внедрении унифицированной информационной системы по управлению процессом обучения. 

Анализ существующих информационных систем показывает, что все они разрозненные, не существует единого протокола обмена и хранения данных учебного процесса, а поскольку ниша свободна, есть смысл производить разработку программного обеспечения, направленного на автоматизацию данной области.

\subsection{Учебный процесс}
\label{sec:domain:study_process}

В данном подразделе описан учебный процесс с точки зрения основных действующих лиц и описаны их деятельности, которые оказывают прямое влияние на архитектуру информационной системы и, в частности, базы данных.

В учебном процессе участвует большое число разнообразных актёров, которые осуществляют огромное число разнообразных деятельностей. Поэтому необходимо обозначить, какие именно деятельности будут автоматизированы в разрабатываемом программном средстве и отражены в проектируемой базе данных.

Два самых важных актёра предметной области, те, для кого учебный процесс первоначально и был придуман, -- это студент и преподаватель. Преподаватель выступает в роли источника знаний, опыта и способен передавать их студенту как приёмнику.

Один и тот же человек может выступать в ролях нескольких актёров одновременно. Например, один студент в одно и то же время обучается в нескольких группах на разных факультетах, а после окончания университета его приглашают преподавать некоторые занятия; через некоторое время этот человек поступает в магистратуру, параллельно продолжает преподавать. Это говорит о том, что в одно и то же время одному человеку может соответствовать некоторый набор ролей, не обязательно различных, и что этот набор может изменяться со временем.

Студенты распределяются по группам. Так как у одного человека может быть несколько ролей студента, то значит у одному человеку может соответствовать несколько групп. 

Группы составляются исходя из принципов различения специальностей. Выпускников какой-либо специальности готовят выпускающие кафедры.

Один университет состоит из факультетов. Факультет управляется деканатом. Работники деканата занимаются студентами, их распределением по группам, составлением расписания занятий. 

Факультеты состоят из кафедр. Кафедры управляются заведующими кафедр и их заместителями. Преподаватели приписываются к какой-либо кафедре. 

Одна из основных деятельностей студентов и преподавателей -- это посещать занятия. Занятия составляются диспетчерской службой согласно учебным планам. Учебный план -- это таблица, в которой строкам соответствуют все дисциплины и курсы, которые необходимо изучить студентам какой-либо специальности и специализации за всё время обучения, а столбцам -- распределение этих дисциплин по всем семестрам обучения и число часов, которое необходимо затратить на их изучение, в распределении по семестрам. Таким образом, некоторой группе, студенты которой обучаются по некоторой специальности, в некотором семестре можно сопоставить определенный набор предметов (дисциплин, курсов). Учебный план рассматривается и утверждается Советом университета. На основание учебного плана деканат составляет расписание, а в учебном плане прописано, какая кафедра у данной специальности будет преподавать тот или иной предмет. Преподаватели назначаются на преподавание предметов на заседаниях кафедры.

В рамках одной дисциплины у студентов и преподавателей могут быть занятия различных видов: лекции, лабораторные, практические занятия (список открытый, возможно в будущем в процессе улучшения обучения будут появляться новые виды). Нет такого вида занятий, который должен быть у предмета обязательно, то есть возможны варианты, когда в рамках курса только читаются лекции, только проводятся практические занятия. Конечно, возможны варианты, когда проводятся все или только часть всех видов занятий. 

Нет строгой связи между тем, что какой-то вид занятий у одной группы должен вести только один преподаватель. Может такое быть, что один вид преподаёт сразу несколько преподавателей (тогда обычно возникает понятие подгруппы студентов) или что какому-то виду занятий преподаватель не сопоставлен (такое может быть, например, в случаях, когда обязанность проводить занятия перенесена на специалистов какого-то предприятия; может быть заранее неизвестно, какой именно специалист придёт в качестве преподавателя на конкретную пару). 

Курс обычно заканчивается каким-то видом контроля знаний: экзаменом, зачётом, дифференцированным зачётом и т.д. Как правило, лектор предмета проводит этот итоговый контроль знаний. Однако, как было сказано ранее, в рамках курса лекций может быть и не предусмотрено, и итоговый контроль знаний будет проводить какой-то другой преподаватель. Результаты итогового контроля знаний заносятся в ведомость преподавателя и зачётную книжку студента.

В рамках учебного процесса студенты выполняют определенные задания и деятельности. Данные задания, как правило, являются обязательными для выполнения и оцениваются. Задания бывают нескольких видов:
\begin{itemize}
	\item Привязанные к определенному виду предмета. Например, лабораторные работы, практические задания, контрольные работы, типовые расчёты и т.д. 
	\item Привязанные к определенному предмету. Это, например, ранее рассмотренный итоговый контроль знаний, курсовое проектирование и т.д.
	\item Непривязанные к какому-либо предмету. Сюда можно отнести различные учебные практики, дипломное проектирование и т.д.
\end{itemize}

Преподаватель может не ограничиваться приведенными видами заданий и использовать в процессе обучения иные виды. Как правило студенту для выполнения какого-либо задания назначается руководитель (возможно, из преподавательского состава).

Некоторые виды заданий выдаются массово, и преподаватели используют различные способы доведения их условий до студентов. Некоторое время назад широко использовался способ составления преподавателем лабораторных практикумов -- специальных книг с краткой теорией, необходимой для выполнения, и самих условий заданий. Однако такой способ можно использовать только тогда, когда курсы и задания в их рамках не меняются в течение долгого времени. Кроме того, этот способ требует затрат со стороны университета на печать таких книг. В настоящее время, когда курсы и дисциплины постоянно менются, адаптируясь к условиям всё ускоряющейся жизни, такой способ практически неприменим. Преподаватели постоянно изменяют и дорабатывают условия заданий, чтобы они соответствовали заданиям, которые встретят студенты в процессе работы после окончания университета. Кроме того, всё большее число преподавателей предпочитает использовать электронные ресурсы и интернет для распространения заданий: это намного более практичный и удобный способ. Единственный минус состоит в том, что преподавателям приходится выбирать какой-либо способ коммуникации со студентами из огромного их числа.

Помимо условий заданий, преподаватели часто выдают студентам список книг, который должен помочь справиться с заданиями, глубже понять материал курса. Кроме того, преподаватели выдают готовые конспекты лекций, материалы презентаций. И здесь снова возникает проблема доведения этих материалов до студентов, каждый преподаватель решает эту проблему своим путём. Это создает сложности для студентов, потому что они могут и не знать, что такие материалы доступны, их можно скачать, просмотреть.

Как уже упоминалось, у участников учебного процесса нет возможности сразу получить календарь занятий. Многие решают эту проблему, используя сторонние сервисы. Однако это не самый удобное решение, так как его приходится целиком заполнять вручную.

Еще один аспект, который никак не оцифрован, -- это ведение списков студентов. В настоящее время, когда группа приходит на первое занятие, староста группы подаёт такой список, а преподаватель использует его в различных целях: для учёта посещаемости студентов, для учёта выполнения ими заданий и т.д. Некоторые преподаватели используют электронные таблицы для этого, но, опять же, ведение этих таблиц полностью зависит от преподавателя. Часто преподаватели, которые проводят итоговый контроль (зачёт, экзамен) просят, чтобы списки с лабораторных и практических занятий каким-то образом попали к ним (через студентов, старосту или напрямую от преподавателя к преподавателю). 

В БГУИР на многих кафедрах широко применяется следующий способ обучения: преподаватель выдаёт задание, студент его дома выполняет, затем приходит на пару к преподавателю и сдаёт. Такой подход часто неудобен для студентов (особенно последних курсов). Кроме того, он приводит к большим очередям. Иногда у студентов получается организовать очередь, но иногда это приводит к конфликтам.

Рассмотренные аспекты предметной области оказывают прямое влияние на разрабатываемое программное средство и проектируемую базу данных. Кроме того, они могут быть автоматизированы, что намного повысит удобство следования учебному процессу у студентов и преподавателей.
