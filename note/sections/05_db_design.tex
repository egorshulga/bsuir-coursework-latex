\section{Проектирование базы данных}
\label{sec:db_design}

Разрабатываемая база данных будет использоваться программным продуктом, целью создания которого является автоматизация учебного процесса. 

Учебный процесс происходит при участии действующих лиц, которые выполняют в его рамках огромное число деятельностей. На первом этапе разработки цель автоматизировать абсолютно все деятельности не стоит (возможно, с каждой последующей версией будут реализовываться новые модули для новых видов деятельности). Для первой версии программного продукта для реализации были выбраны следующие аспекты деятельности:
\begin{itemize}
	\item управление студентами и группами студентов;
	\item управление преподавателями;
	\item предметы;
	\item виды предметов;
	\item календарь событий учебного процесса;
	\item задания для студентов;
	\item контроль знаний;
	\item управление списками преподавателей.
\end{itemize}

Система предполагает управление большим количеством важной личной информации студентов и преподавателей, поэтому для реализации системы пользователей и их прав доступа в модели БД предусмотрены таблицы users и users\_authentications, которые предназначены для хранения информации о пользователях и их средствах аутентификации соответственно.

Таблица user\_assignments представляет собой центр системы ролей: она предназначается для хранения ролей пользователей и истории их назначения. Само имя роли задается в этой таблице в виде типа-перечисления, потому что количество ролей должно быть строго задано, так как от них зависит остальная структура базы. Предусмотрены следующие роли:
\begin{itemize}
	\item student -- студент;
	\item lecturer -- преподаватель;
	\item university\_admin -- администратор университета;
	\item faculty\_admin -- администратор факультета;
	\item chair\_admin -- администратор кафедры.
\end{itemize}

Предполагается, что администраторы различных уровней будут обладать различными обязанностями. Администратор университета отвечает за общую структуру университета и вдобавок устанавливает семестры учебного процесса (таблица semesters). Администратор факультета отвечает за подтверждение ролей студентов и создаёт расписание для всех групп факультета. Администратор кафедры отвечает за подтверждение ролей преподавателей. Кроме того, все администраторы получают право управления досок университета, факультета, кафедры соответственно.

Система предполагает наличие многих университетов, соответствующих им факультетов и, в свою очередь, соответствующих им кафедр. Для хранения данной структуры предназначены таблицы universities, faculties, chairs. 

Для реализации досок объявлений университета, факультета, кафедры предназначаются таблицы university\_news, faculty\_news, chair\_news. Как было упомянуто выше, управление ими возлагается на администраторов различных уровней. Новости на досках объявления могут иметь прикрепления в виде файлов, поэтому таблицы новостей имеют связь многие-ко-многим с таблицей файлов, которая будет рассмотрена ниже.

Для хранения специальностей предусмотрена таблица specialties. Как было описано в разделе~\ref{sec:domain:study_process}, специальности соответствуют кафедрам в отношении многие-к-одному. Специальность -- атрибут, относящийся к группам студентов (таблица groups). Одной группе соответствует номер группы, однако этот признак не является потенциальным ключом, так как система формирования номеров не предусматривает уникальности: в БГУИР номера групп повторяются с периодом в 10 лет, поэтому потенциальным ключом является комбинация номера группы и года ее поступления.

Для ролей студентов предусмотрена таблица students. Каждому студенту соответствует номер, устанавливаемый деканатом. 

Студент связан с таблицей назначения ролей в отношении один-ко-многим по следующей причине: один студент может несколько раз поступать в университет, -- данное решение избавляет от необходимости каждый раз создавать нового студента. Кроме того, студент в отношении многие-ко-многим связан с группами.

Для преподавателей предусмотрена таблицы lecturers. Исходя из условий предметной области, каждый преподаватель приписан к какой-либо кафедре, кроме того, ему соответствует номер работника и название должности.

Студенты изучают предметы (таблица subjects), однако один предмет может преподаваться в нескольких семестрах. Для фиксации этого факта предусмотрена таблица subjects\_in\_semesters. 

В рамках конкретного курса может быть несколько типов занятий, которые ведут разные преподаватели. Этот факт устанавливается в таблице subject\_activities. Кроме того, данная таблица связана с таблицей групп отношением многие-ко-многим. А для самих типов занятий предусмотрена таблица subject\_activity\_types.

В результате анализа предметной области было выявлено три вида заданий, которые могут выполнять студенты: задания в рамках курса, в рамках какого-то вида занятий курса и безотносительно курсов. Первый тип заданий представлен в таблице subject\_activities, которая была упомянута выше. Для второго типа заданий предусмотрена таблица subject\_activity\_tasks. С ней связана таблица subject\_activity\_task\_types, которая предназначена для типов таких заданий. В самой таблице заданий хранится условие задания. Условие может иметь прикрепления в виде файлов, поэтому предусмотрена связь многие-ко-многим с таблицей файлов. Для третьего типа заданий предусмотрены таблицы students\_activities, activities и activity\_types.

Для материалов преподавателей предусмотрена таблица textbooks, которая связана с таблицей файлов.

В рамках какого-то типа предмета курса преподаватель может вести списки студентов. Для этого предусмотрена таблица students\_list. Однако, не все студенты могут быть зарегистрированы в системе, поэтому с помощью таблиц non\_registered\_students и маршрутизирующей таблицы students\_in\_lists реализуется возможность преподавателю добавить фамилию человека себе в список в случае, если у этого человека нет зарегистрированной в системе учетной записи.

Списки студентов используются при проверке посещаемости и для отслеживания прогресса сдачи заданий, для чего и предназначены таблицы students\_attendances и students\_tasks соответственно.

В проектируемой базе данных предусматривается возможность нескольких попыток сдачи заданий. Для этой цели и предусмотрена таблица попыток сдачи task\_defence\_attempts.

Для итогового контроля знаний (экзамена, зачёта) предусмотрена таблица examinations и таблицы, которые связывают конкретный экзамен со студентами. У студентов есть возможность сдавать экзамен несколько раз, для чего предусмотрена таблица examination\_attempts.

Все деятельности учебного процесса имеют своё отражение в виде событий календаря. Для всех событий календаря предусмотрена таблица activity\_events. Любое событие имеет время (представлено в виде атрибутов таблицы) и место (представлено в виде таблиц auditoriums и buildings).

Для пользователей системы было бы удобно, если бы система оповещала их о любых изменениях в расписании, о сдаче заданий, о иных событиях. Для этих целей и предусмотрена таблица notifications.

Для пользователей предусмотрена возможность обмена личными сообщениями. Для самих сообщений предназначена таблица messages.

Для учета загружаемых в систему файлов предусмотрена таблица files и таблицы отношений многие-ко-многим с другими сущностями системы. Так как в MySQL не предусмотрен тип файла, то эмулировать его приходится самостоятельно. В таблице файлов предусмотрены следующие атрибуты: file\_name (исходное имя файла) и file\_path (имя файла при хранении на сервере). Предполагается, что при загрузке файлов во избежание коллизии имён они будут переименовываться (в качестве имени можно использовать, например, хеш файла).

Скрипт генерирования базы данных приведен в приложении А. Модель базы данных приведена на графическом материале ГУИР.351005-01 ПЛ.
