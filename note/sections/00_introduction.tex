\sectioncentered*{Введение}
\addcontentsline{toc}{section}{Введение}
\label{sec:introduction}

Информация -- сведения, воспринимаемые человеком или специальными устройствами как отражение фактов материального мира в процессе коммуникации \cite{gost_7.0}. Хотя информация должна обрести некоторую форму представления (то есть превратиться в данные), чтобы ей можно было обмениваться, информация есть в первую очередь интерпретация (смысл) такого представления \cite{iso_iec_vocabulary}. Понятие информации рассматривалось еще античными философами.

В современном мире информация представляет собой один из важнейших ресурсов и, в то же время, одну из движущих сил развития человеческого общества. Информационные процессы, происходящие в материальном мире, живой природе и человеческом обществе изучаются (или, по крайней мере, учитываются) всеми научными дисциплинами от философии до маркетинга \cite{wiki_information}.

Данные — это поддающееся многократной интерпретации представление информации в формализованном виде, пригодном для передачи, связи, или обработки \cite{iso_iec_it_vocabulary}. Практически все современные информационные системы основаны на некотором манипулировании данными. Как правило, эти данные имеют достаточно сложную структуру и, вдобавок, имеют большой объём. Кроме того, для таких информационных систем большое значение имеет возможность сохранения данных между последовательными запусками программы.

На первых этапах развития информационных систем данные хранились бесструктурно: сначала на внешних физических носителях (например, перфокартах), затем в текстовых или бинарных файлах \cite{habr_db_tutorial}. 

Следующий важный этап связан с появлением в начале 1970-х реляционной модели данных, благодаря работам Эдгара Кодда \cite{wiki_db}. Некоторое время развитие баз данных проходило в стенах университетов, но уже в 1974 году IBM выпустило первую версию реляционной системы управления базами данных System R. Для этой системы был разработан новый язык программирования -- SEQUEL, позже получивший название SQL. 

Немного позже на основе имеющихся наработок IBM выпустило систему управления базами данных DB2; компания Oracle выпустила свою версию СУБД; компания Microsoft выпустила первую версию своей СУДБ -- Microsoft SQL Server. Затем в 1990-е годы наступил этап бурного развития настольных компьютеров: начали появляться версии СУБД, ориентированные и на них, а не на мейнфреймы или мощные серверные станции: например, в 1995 году была выпущена первая версия MySQL.

Вызывает удивление тот факт, что системы, первые версии которых выпущены еще 30 и 40 лет назад, по-прежнему востребованы и широко применяются по всему миру, как и принципы, применяемые при проектировании.
