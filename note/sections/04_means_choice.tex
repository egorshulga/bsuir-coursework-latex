\section{Используемые технологии} 
\label{sec:practice:menas_choice}

Перед проектированием базы данных на даталогическом уровне необходимо осуществить выбор используемых технологий. Выбор технологий является важным предварительным этапом разработки сложных информационных систем.
Платформа, на которой будет реализована система, заслуживает большого внимания, так как исследования показали, что выбор языка программирования влияет на производительность труда программистов и качество создаваемого ими кода~\cite[c.~59]{mcconnell_2005}.

Ниже перечислены некоторые факторы, повлиявшие на выбор технологий:
\begin{itemize}
	\item Должна быть возможность развёртывать проектируемую БД на ПК с операционной системой Windows~7 и более новых версиях системы.
	\item Среди различных систем управления базами данных имеющийся программист лучше всего знаком с СУБД MySQL и CASE средством Sparx Enterprise Architect.
	\item Дальнейшей поддержкой проекта, возможно, будут заниматься разработчики, не принимавшие участие в 	выпуске первой версии.
\end{itemize}

MySQL -- свободная реляционная система управления базами данных. Разработку и поддержку MySQL осуществляет корпорация Oracle. Продукт распространяется как под GNU General Public License, так и под собственной коммерческой лицензией, что позволяет легально использовать данную СУБД вместе с программным обеспечением практически любого вида лицензирования. MySQL является решением для малых и средних приложений~\cite{wiki_mysql}. 

Sparx Systems Enterprise Architect -- средство визуального моделирования и проектирования. Данная платформа поддерживает проектирование и построение программных систем, моделирование бизнес-процессов, проектирование баз данных (как на инфологическом, так и на даталогическом уровнях) и многие другие виды проектирования. В аспекте проектирования баз данных Enterprise Architect поддерживает следующие виды нотаций: DDL, ERD, IDEF1X, UML, -- и большое число СУБД, среди которых: DB2, MS Access, MS SQL Server, MySQL, SQLite, Oracle, PostgreSQL и многие другие~\cite{wiki_sparx_ea}. Одним из преимуществ данного CASE средства является возможность обратного проектирования: построение модели по уже развернутой базе данных, построение диаграммы классов по исходному коду приложения и т.д.

Основываясь на опыте работы имеющихся программистов проектировать БД целесообразно с применением среды разработки Sparx Enterprise Architect, позволяющей осуществлять проектирование базы данных в нотации UML 2.1, при этом целевой системой управления базами данных целесообразно выбрать MySQL.
